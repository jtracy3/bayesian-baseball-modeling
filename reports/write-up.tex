
\documentclass[11pt]{article}

%%%%%%%%%%%%%%%%%%%
% Page Layout
%%%%%%%%%%%%%%%%%%%

\setlength{\paperwidth}{8.5in} \setlength{\paperheight}{11in}
\setlength{\marginparwidth}{0in} \setlength{\marginparsep}{0in}
\setlength{\oddsidemargin}{0in} \setlength{\evensidemargin}{0in}
\setlength{\textwidth}{6.5in} \setlength{\topmargin}{-0.5in}
\setlength{\textheight}{9in}

%%%%%%%%%%%%%%%%%%%%%%%%%%%%%%%%%%%
% Include Packages and Style Files
%%%%%%%%%%%%%%%%%%%%%%%%%%%%%%%%%%%

\usepackage[english]{babel}
\usepackage{amsmath,amssymb,amsthm}
\usepackage{enumerate}
%\usepackage[pdftex]{graphicx,color}

%%%%%%%%%%%%%%%%%%%%%%%%%%%%%%
% Define theorem environments
%%%%%%%%%%%%%%%%%%%%%%%%%%%%%%

\newtheorem{theorem}{Theorem}[section]
\newtheorem{proposition}[theorem]{Proposition}
\newtheorem{lemma}[theorem]{Lemma}
\newtheorem{corollary}[theorem]{Corollary}
\newtheorem{claim}[theorem]{Claim}
\newtheorem{question}[theorem]{Question}
\newtheorem{conjecture}[theorem]{Conjecture}

\theoremstyle{definition}
\newtheorem{definition}[theorem]{Definition}
\newtheorem{example}[theorem]{Example}
\newtheorem*{remark}{Remark}

%%%%%%%%%%%%%%%%%%%%%%
% Define new commands
%%%%%%%%%%%%%%%%%%%%%%

\newcommand{\R}{\mathbb{R}}
\newcommand{\mb}[1]{\mathbf{#1}}

\begin{document}

\title{Hierarchical Bayesian Modeling for Home Run Totals in the MLB}
\author{Josh Tracy}
\date{}



\maketitle


\section{Introduction}



\section{Data}

\subsection{Source of Data}

\subsection{Cleaning}


\section{Model}


\section{Markov Chain Monte Carlo}

Markov chains are discrete sequences of numbers such that $p(x_n | x_{n-1}, x_{n-2}, ...x_2, x_1) = p(x_n | x_{n-1})$. Markov chains are named such because they exhibit the Markov property. Derived by Russian mathematician Andrey Markov, the Markov property refers to the memorylessness of a stochastic procces. That is, the probability of state $x_n$ is dependent on only the previous state $x_{n-1}$. You only need to know the current state to predict the subsequent state. Knowing the past history of states won't provide additional information regarding any future state. 

\subsection{Metropolis-Hastings Algorithm}


\section{Results}

\section{Conclusions}

\subsection{Future Modifications}


\section*{Bibliography}



\end{document}
